\documentclass[a4paper]{article}

\usepackage{etoolbox}
\usepackage{amsmath, amsthm, amssymb}
\usepackage{a4wide}
\usepackage{setspace}
\onehalfspacing

\RequirePackage[dvipsnames]{xcolor}
\RequirePackage{dsfont}

\usepackage[%
  colorlinks = true,
  citecolor  = RoyalBlue,
  linkcolor  = RoyalBlue,
  urlcolor   = RoyalBlue,
  unicode,
  ]{hyperref}
\usepackage[capitalize]{cleveref}


\usepackage[%
  autocite     = plain,
  backend      = biber,
  doi          = true,
  url          = true,
  giveninits   = true,
  hyperref     = true,
  maxbibnames  = 99,
  maxcitenames = 99,
  sortcites    = true,
  style        = numeric,
  ]{biblatex}
\addbibresource{refs.bib}


\theoremstyle{plain}

\newtheorem{theorem}{Theorem}
\newtheorem{proposition}[theorem]{Proposition}
\newtheorem{lemma}[theorem]{Lemma}

\theoremstyle{definition}
\newtheorem{example}[theorem]{Example}
\newtheorem{definition}[theorem]{Definition}
\newtheorem{remark}[theorem]{Remark}
\newtheorem{convention}[theorem]{Convention}
\newtheorem{problem}[theorem]{Problem}

\newcommand{\liff}{\leftrightarrow}
\newcommand{\iffdef}{\stackrel{\mathrm{def}}{\iff}} 
\newcommand{\To}{\Rightarrow}  
\newcommand{\onto}{\twoheadrightarrow}
\newcommand{\plainRel}{\to}
\newcommand{\rel}[1]{\stackrel{#1}{\plainRel}}
\newcommand{\plainrRel}{\leftarrow}
\newcommand{\rrel}[1]{\stackrel{#1}{\plainrRel}}

\newcommand{\mbb}[1]{\mathbb{#1}}
\newcommand{\mbf}[1]{\mathbf{#1}}
\newcommand{\mr}[1]{\mathrm{#1}}
\newcommand{\mc}[1]{\mathcal{#1}}
\newcommand{\ms}[1]{\mathsf{#1}}

\DeclareMathOperator*{\dom}{\mathrm{dom}}
\DeclareMathOperator*{\im}{\mathrm{im}} 
\newcommand{\bN}{\mathbb{N}}
\newcommand{\cP}{\mathcal{P}}
\newcommand{\op}{{\mathrm{op}}}
\newcommand{\sem}[1]{\llbracket{#1}\rrbracket}
\newcommand{\gen}[1]{\langle #1 \rangle}
\newcommand{\isdef}{\stackrel{\mathrm{def}}{=}}
\newcommand{\Reg}{\mathsf{Reg}}
\DeclareMathOperator*{\Spec}{\mathsf{Spec}}
\DeclareMathOperator*{\Clp}{\mathsf{Clp}}
\newcommand{\id}{\mathsf{id}}
\newcommand{\BoolAlg}{\mbf{BoolAlg}}
\newcommand{\BoolSp}{\mbf{BoolSp}}

\newcommand{\pro}{\mathsf{pro}}
\newcommand{\set}{\mathsf{set}}
\newcommand{\disc}{\mathsf{disc}}
\renewcommand{\c}{\mathsf{c}}

\newcommand{\zer}{0}
\newcommand{\one}{1}

\newcommand{\old}[1]{}


\title{Stone duality for Boolean algebras}
\author{Sam van Gool}

\begin{document}
\maketitle

The aim of this note is to give a detailed proof of Stone duality for Boolean
algebras \cite{Sto1937BA} to facilitate its formalization in Mathlib. Note that
Stone actually proved something more general for bounded distributive lattices
\cite{Sto1937}. (Below, I will cut everything into very small numbered lemmas
and definitions which is not how I usually like to write math, but it seemed
like it might be useful for organizing the formalization.)

\section{Definitions and statement}
I first recall some definitions from Mathlib. A subset $S$ of a topological
space is \emph{preconnected} if, whenever $U$ and $V$ are open subsets such
that $S \subseteq U \cup V$, $S \cap U \neq \emptyset$ and $S \cap V \neq
\emptyset$, we have $S \cap U \cap V \neq \emptyset$. A topological space is
\emph{totally disconnected} if any preconnected subset is a subsingleton. 

A Boolean space is a compact Hausdorff totally disconnected space.%
\footnote{What I call Boolean space is called \texttt{Profinite} in Mathlib and
is sometimes called `Stone space' by other mathematicians.} The category
$\BoolSp$ of Boolean spaces is the full subcategory of topological spaces based
on Boolean spaces.

A \emph{Boolean algebra} is by definition a bounded distributive lattice with
complements. This means that it is a structure of the form $(A, \vee, \wedge,
\zer, \one, \neg)$ satisfying a number of axioms. We do not really care what
the axioms are precisely. What is important to know is that, for any set $X$,
the structure $(\mathcal{P}(X), \cup, \cap, \emptyset, X, (-)^{\c})$ is a
Boolean algebra (I believe and hope that this is in Mathlib, if not, it should
be.) In particular, I write $\mbf{2} = \{ \zer, \one \}$ for the unique
two-element Boolean algebra; it can be seen that it is a Boolean algebra
because it is the power set algebra of a singleton set. A Boolean algebra
\emph{homomorphism} is a structure-preserving function between Boolean
algebras. In fact, to be a homomorphism, it suffices to preserve $\vee$,
$\zer$, and $\neg$, because the operations $\one$ and $\wedge$ are
term-definable from those. In particular, for a subset $S$ of a Boolean algebra
$A$ to be a subalgebra, it suffices for $S$ to contain $\zer$, and be closed
under the operations $\vee$ and $\neg$. An \emph{embedding} of Boolean algebras
is the same thing as an injective homomorphism. An \emph{ideal} of a Boolean
algebra is a subset $I$ which contains $\zer$ and is such that $a \vee b \in I$
if, and only if, both $a \in I$ and $b \in I$ (this is equivalent to the usual
definition for rings but more convenient to work with). An ideal is
\emph{proper} if it does not contain $\one$. The \emph{partial order} on a
Boolean algebra $A$ can be defined by $a \leq b$ iff $a \vee b = b$, or
equivalently  iff $a \wedge b = a$.

We will need the following fundamental fact about Boolean algebras which is
sometimes called the ultrafilter principle. It can be deduced from the
ring-theoretic fact (probably in Mathlib) that any non-unit element of a ring
is in some maximal ideal.
\begin{lemma}\label{ultrafilterprinciple}
    Let $A$ be a Boolean algebra. For any $a \in A \setminus \{\one\}$, 
    there exists a homomorphism $x \colon A \to \mbf{2}$ such that $x(a) =
    \zer$.
\end{lemma}
\begin{proof}
    By Zorn's lemma, let $I$ be a maximal element of the set of proper ideals of $A$
    which contain $a$. Define $x(b) = \zer$ iff $b \in I$. Clearly, $x(a) =
    \zer$; we need to check that $x$ is a homomorphism. 
The equalities $x(\zer) = \zer$ and $x(b \vee b') = x(b) \vee x(b')$ are
    easy to check from the defining properties of an ideal. To see that $x(\neg
    b) = \neg x(b)$ for any $b \in A$, the crucial observation is that if $b
    \not\in I$ and $\neg b \not\in I$, then it is possible to enlarge $I$ by
    adding $b$ to it, while staying proper. By maximality of $I$ this is
    impossible.  We thus get that, for any $b \in A$, one of $b$ and $\neg b$
    must be in $I$, and they can never be both in $I$, since that would give
    $\one = b \vee \neg b$ in $I$, contradicting that $I$ is proper. It now
    follows from the definitions that $x(\neg b) = \neg x(b)$.
\end{proof}
The statement we want to prove is the following:
\begin{theorem}\label{BABoolSpdualeq}
    The categories $\BoolAlg$ and $\BoolSp$ are dually equivalent.
\end{theorem}
It will also be useful and interesting to actually have concrete definitions of
the dual equivalence functors both ways, so we do that first.
\begin{remark}
An alternative proof that is shorter on paper but does not give explicit access
to the definition of the dual equivalence functors would be to use the
fact, which is 
\href{https://leanprover-community.github.io/mathlib4_docs/Mathlib/CategoryTheory/Equivalence.html#CategoryTheory.Equivalence.ofFullyFaithfullyEssSurj}{already
in Mathlib}, that any fully faithful essentially surjective functor is part
of an equivalence. However, the inverse part of the equivalence that is
produced by that fact only exists thanks to an application of choice, and it
seems like it would be hard to work with. But I could be wrong about this, and
implementing this road towards the proof might be an interesting alternative experiment. It
has the advantage of avoiding having to get into the weeds of natural
transformation arguments as I will need to do below.
\end{remark}

\section{The functors}
We define the two functors that make up Stone duality for Boolean algebras.

\subsection{From spaces to algebras}
If $X$ is a Boolean space, write $\Clp(X)$ for the set of clopen
subsets of $X$. 

\begin{proposition}\label{clpwelldef}
    $\Clp(X)$ is a Boolean subalgebra of the Boolean algebra $\mathcal{P}(X)$.
\end{proposition}
\begin{proof}
Finite unions and complements of clopen sets are clopen, and the empty set is
clopen.
\end{proof} 
\begin{proposition}\label{clpmorphism}
    If $f \colon X \to Y$ is a
continuous function between Boolean spaces, then $f^{-1} \colon \Clp(Y) \to \Clp(X)$ is a Boolean
algebra homomorphism.
\end{proposition}
\begin{proof}
    $f^{-1}(\emptyset) = \emptyset$, $f^{-1}(K^\c) =
f^{-1}(K)^\c$, and $f^{-1}(K_1 \cup K_2) = f^{-1}(K_1) \cup f^{-1}(K_2)$ for
any clopens $K, K_1, K_2$ of $Y$. 
\end{proof}
One sometimes writes $\Clp(f)$ for
$f^{-1}$ but the notation is a bit heavy.

\begin{proposition}
    The assignments $X \mapsto \Clp(X)$ and $f \mapsto f^{-1}$ give a
contravariant functor $\BoolSp \to \BoolAlg$.
\end{proposition}
\begin{proof}
    The assignments are well-defined by \cref{clpwelldef} and
    \cref{clpmorphism}.
It is a contravariant functor because $\ms{id}_X^{-1}(K) = K$ and $(f
\circ g)^{-1}(K) = g^{-1}(f^{-1}(K))$ for any clopen $K$.
\end{proof}

\subsection{From algebras to spaces}
If $A$ is a Boolean algebra, write
$\Spec(A)$ for the set of homomorphisms from $A$ to $\mbf{2}$. 
We equip $\Spec(A)$ with the subspace topology induced by the $|A|$-fold power
$\mbf{2}^{A}$, where $\mbf{2}$ has the discrete topology.
Define, for any $a \in A$,
\[ U_a \isdef \{ x \in \Spec(A) \ \mid \ x(a) = \one \} \ . \]

\begin{lemma}\label{Uacomp}
    For any $a \in A$, the complement of $U_a$ is equal to $U_{\neg a}$.
\end{lemma}
\begin{proof}
Note that 
$x \in (U_a)^\c \iff x(a) \neq \one \iff x(a) = \zer \iff x(\neg a) =
\one \iff x \in U_{\neg a} \ .$ 
\end{proof}
\begin{lemma}\label{Uaclp}
    For any $a \in A$, the set $U_a$ is clopen in $\Spec(A)$.
\end{lemma}
\begin{proof}
    For any $a \in A$, the set $U_a$ is open because it is $\pi_a^{-1}(\one)$ where $\pi_a$
    denotes the restriction of the projection function from $\Spec(A)$ to
    $\mbf{2}$ on the coordinate $a$, and this projection function is continuous
    by the definitions of product and subspace topology. The set $U_a$ is
    closed because its complement is equal to $U_{\neg a}$, which we already
    showed is open.
\end{proof}
\begin{proposition}\label{Specbasis}
The topology on $\Spec(A)$ is generated by the basis
$(U_a)_{a \in A}$ of clopen sets. 
\end{proposition}
\begin{proof}
    In general, the topology on a product $\prod_{i\in I} X_i$ is generated by
    the sets $\pi_i^{-1}(U)$ where $i \in I$ and $U \subseteq X_i$ ranges over
    a basis for the open sets of $X_i$. Now in the case of $\mbf{2}^A$, this
    implies that the topology has as a basis the sets $\pi_a^{-1}(\zer)$ and
    $\pi_a^{-1}(\one)$, as $a$ ranges over the elements of $A$. But
    \cref{Uacomp} gives that $\pi_a^{-1}(\zer) = \pi_{\neg a}^{-1}(\one)$, so
    it actually suffices to take the sets $U_a = \pi_a^{-1}(\one)$ as $a$
    ranges over the elements of $A$.
\end{proof}
\begin{theorem}
    The topological space $\Spec(A)$ is a Boolean space.
\end{theorem}
\begin{proof}
    If $x, y \in \Spec(A)$ are distinct, then there is an element $a \in A$
    such that $x(a) \neq y(a)$. Without loss, say that $x(a) = \one$ and $y(a)
    = \zer$. Then $x \in U_a$ and $y \not\in U_a$. Thus, any two distinct
    elements are separated by a clopen set. This suffices to conclude that
    $\Spec(A)$ is totally disconnected by
    \href{https://leanprover-community.github.io/mathlib4_docs/Mathlib/Topology/Connected/TotallyDisconnected.html#isTotallyDisconnected_of_isClopen_set}{
    a result that is in Mathlib already}, and it also clearly implies that the
    space is Hausdorff. It remains to prove that $\Spec(A)$ is compact. This is
    where a weak form of axiom of choice must be used. 
    We show that the set of homomorphisms
    is closed as a subspace of $\mbf{2}^{|A|}$, and it is therefore compact,
    since $\mbf{2}^{|A|}$ is compact by Tychonoff's Theorem. In order to see
    that $\Spec(A)$ is closed in $\mbf{2}^{|A|}$, notice that
    \[ \Spec(A) = \bigcap_{a,b\in A} J_{a,b} \cap \bigcap_{a \in A} N_a \cap Z
    \ , \]
    where $J_{a,b} \isdef \{ x \in \mbf{2}^A \ : \ x(a \vee b) = x(a) \vee x(b) \},$
    $N_a \isdef  \{x \in \mbf{2}^A \ : \ x(\neg a) = \neg x(a) \},$ $Z \isdef \{ x
    \in \mbf{2}^A \ : \ x(\zer) = \zer\}.$ Each of these sets is clopen because its
    definition only depends on a finite number of coordinates of $x$. For example, to
    spell this out a bit more,
    for any $a, b \in A$,
    \[ J_{a,b} = 
        (\pi_{a \vee b}^{-1}(\zer) \cap \pi_{a}^{-1}(\zer) \cap
        \pi_{b}^{-1}(\zer)) \cup (\pi_{a \vee b}^{-1}(\one) \cap
    (\pi_a^{-1}(\one) \cup \pi_b^{-1}(\one))) \ , \]
    where $\pi_c \colon \mbf{2}^A \to \mbf{2}$ is the projection onto the $c$
    coordinate.\qedhere

    \old{Here is a more concrete proof. By a well-known fact of topology that is
    hopefully in Mathlib, it suffices to prove the finite subcovering property
    with respect to the basis.  Let $T \subseteq A$ be a collection of elements
    with the property that $\bigcup_{a \in T} U_a = \Spec(A)$. Let $I$ be the
    ideal of $A$ generated by $T$. Here ideal means the same as in ring theory,
    but a more convenient definition is the following. This ideal can be
    described concretely as $$I = \{ b \in A \ \mid \ \text{ there is a finite
    } F \subseteq T \text{ such that } b \leq \bigvee F\}\ . $$ If we can prove
    that $\one \in I$, then we are done, because we have a finite $F \subseteq
    T$ such that $\bigvee F = \one$, and then, for any $x \in \Spec(A)$, we get
    \[ \one = x(\one) = x(\bigvee F) = \bigvee_{a \in F} x(a) \, \] so that
    $x(a) = \one$ for some $a \in F$.

    Towards a contradiction, suppose that $\one\not\in I$. By an application of
    Zorn's Lemma, there is a maximal ideal $J$ containing $I$ such that still
    $\one\not\in J$. Let $x \isdef \chi_J \colon A \to \mbf{2}$ be the characteristic function of $J$,
    that is, $x(a) = \zer$ if $a \in J$ and $x(a) = \one$ otherwise. One can
    now prove that $x$ is a homomorphism, as follows. By definition $\zer \in
    I$ so $\zer \in J$, so $x(\zer) = \zer$. Also, in any ideal, we have that
    $a \vee b \in I$ if, and only if, $a \in I$ and $b \in I$. This shows that
    $x(a \vee b) = x(a) \vee x(b)$. Finally, we need to show that $x$ preserves
    negation. Suppose that $x(a) = \one$, so that $a \not\in J$. We need to
    show that $\neg a \in J$. If $\neg a \not\in J$, then $\one$ is not in the
    ideal generated by $J \cup \{a\}$. But this ideal strictly contains $J$,
    contradicting the maximality of $J$. Thus, we must have $x(\neg a) = \zer$.
    Finally, if $x(a) = \zer$, then $a \in J$, and we cannot also have $\neg a
    \in J$, because that would imply $\one = a \vee \neg a$ is in $J$ as well,
contradiction. Thus, $x(\neg a) = \one$.}
\end{proof}
\begin{definition}
    Given a homomorphism $f \colon A \to B$ we define $f_* \colon \Spec(B) \to \Spec(A)$
as the map that sends $y \colon B \to \mbf{2}$ to $y \circ f \colon A \to
\mbf{2}$. 
\end{definition}
This is well-defined: Since $y \circ f$ is the composition of two homomorphisms, it is again a
homomorphism, showing that $f_*$ is a well-defined function. 

\begin{proposition}
    For any Boolean
    algebra homomorphism $f \colon A \to B$, $f_* \colon \Spec(B) \to \Spec(A)$
    is a continuous function.
\end{proposition}
\begin{proof}
    For any $a \in A$ and $y \in
\Spec(B)$, we have
\[ y \in f_*^{-1}(U_a) \iff f_*(y) \in U_a \iff y(f(a)) = \one \iff y \in
U_{f(a)}\ , \]
so that the inverse image $f_*^{-1}(U_a)$ of any basic open set is again a
(basic) open.
\end{proof}

\begin{proposition}
    The assignment $f \mapsto f_*$ is contravariant functorial from $\BoolAlg$
    to $\BoolSp$. 
\end{proposition}
\begin{proof}
    If $f \colon A \to B$ and $g \colon B
\to C$ then for any $y \in \Spec(C)$,
\[ (\id_C)_*(y) = y \circ \id_C = y, \quad (g \circ f)_*(y) = y \circ g \circ f =
f_*(g_*(y)) \ . \qedhere \]
\end{proof}

One sometimes writes $\Spec f$ for $f_*$ but
the notation is a bit heavy.

\section{The equivalence}
\begin{lemma}\label{etahom}
    For any $a, b \in A$ we have $U_a \cup U_b = U_{a \vee b}$, and moreover
    $U_{\zer} = \emptyset$.
\end{lemma}
\begin{proof}
    For any $x \in \Spec(A)$, we have 
    \[ x(a \vee b)= \one \iff x(a) \vee x(b) = \one \iff x(a) = \one \text{ or
    } x(b) = \one \ . \]
    This shows that $U_a \cup U_b = U_{a \vee b}$. To see that $U_{\zer} =
    \emptyset$ just notice that $x(\zer) = \zer \neq \one$.
\end{proof}
\begin{definition}
Let $A$ be a Boolean algebra. We write $\eta_A$ for the Boolean homomorphism
that sends $a \in A$ to $U_a \in \Clp(\Spec(A))$.
\end{definition}
It is indeed a homomorphism by \cref{Uacomp} and \cref{etahom}.
\begin{proposition}
The homomorphism $\eta_A$ is bijective, and thus an isomorphism in $\BoolAlg$.
\end{proposition}
\begin{proof}
    The homomorphism $\eta_A$ is injective: 
    \cref{ultrafilterprinciple} gives that if $U_a = \Spec(A)$ then 
    $a = 1$. Now if $a \neq b$ then $(\neg a \wedge b) \vee (\neg b \wedge a)
    \neq 1$ from which one gets that $\eta(a) \neq \eta(b)$ (this is the
    Boolean algebra version of the usual argument that a ring homomorphism is
    injective if it has zero kernel). 
\end{proof}

\begin{definition}
    Let $X$ be a Boolean space. We write $\epsilon_X$ for the 
    function $X \to \Spec (\Clp(X))$ that sends $x \in X$ to the homomorphism
    $\epsilon_X(x) \colon \Clp(X) \to \mbf{2}$ defined by sending any $K \in \Clp(X)$ to $\one$ if $x \in K$,
    and to $\zer$ otherwise.
\end{definition}
\begin{lemma}\label{epseq}
    Let $X$ be a Boolean space. For any $K \in \Clp(X)$, we have
    $\epsilon_X^{-1}(U_K) = K$.
\end{lemma}
\begin{proof}
    Note that, for any $x \in X$,
    \[ x \in \epsilon_X^{-1}(U_K) \iff \epsilon_X(x) \in U_K \iff
    \epsilon_X(x)(K) = \one \iff x \in K \ . \qedhere \] 
\end{proof}
\begin{proposition}
    The function $\epsilon_X$ is continuous for any Boolean space $X$.
\end{proposition}
\begin{proof}
    By \cref{Specbasis} and a general fact about continuous functions, it
    suffices to check that $\epsilon_X^{-1}(U_K)$ is open for every $K \in
    \Clp(X)$.  But this set is equal to $K$ by \cref{epseq}, and thus open.
\end{proof}

%\begin{proposition}
%    The functor $\Spec$ is full.
%\end{proposition}

%\begin{proposition}
%    The functor $\Spec$ is faithful.
%\end{proposition}

% \section{Characterization of Boolean spaces}
% The following equivalent definition of Boolean spaces is more convenient, and
% partly in Mathlib already. A space $X$ is called \emph{totally separated} if,
% for any distinct $x,y \in X$, there exist disjoint open sets $U, V \subseteq
% X$ such that $x \in U$ and $y \in V$.

% \begin{lemma}\label{lem:equiv-def-Boolean}
% A compact Hausdorff space $X$ is totally disconnected if, and only if, for any
% distinct $x, y \in X$, there exists a clopen set $K \subseteq X$ such that $x
% \in K$ and $y \not\in K$.
% \end{lemma}
% \begin{proof}
%    \href{https://leanprover-community.github.io/mathlib4_docs/Mathlib/Topology/Connected/TotallyDisconnected.html#isTotallyDisconnected_of_isClopen_set}{The
%     sufficiency is in Mathlib already}. The necessity is for example proved as
%     the implication (i) $\Rightarrow$ (ii) of \cite[Thm~II.4.2]{Joh1986} but I
%     think the following (extracted from there) is a bit simpler, given what is
%     already in Mathlib. Suppose that $X$ is a totally disconnected compact
%     Hausdorff space. It suffices to prove that $X$ is totally separated,
%     because it is in Mathlib that
%     \href{https://leanprover-community.github.io/mathlib4_docs/Mathlib/Topology/Connected/TotallyDisconnected.html#exists_isClopen_of_totally_separated}{the
%     separating set in any totally separated space can be chosen to be clopen}.
%     Let $x, y \in X$ be distinct. By total disconnectedness, since $\{x,y\}$ is
%     not connected,
%     \href{https://leanprover-community.github.io/mathlib4_docs/Mathlib/Topology/Connected/Basic.html#isPreconnected_closed_iff}{we
%     can pick} closed sets $F_1, F_2$ such that $\{x,y\} \subseteq F_1 \cup F_2$
%     and each $F_i$ has non-empty intersection with $\{x,y\}$, but $\{x,y\} \cap
%     F_1 \cap F_2 = \emptyset$. Without loss of generality, we have $x \in F_1$,
%     and then $x \not\in F_2$ (because $x$ is not in $F_1 \cap F_2)$, and
%     therefore (because $F_2$ has non-empty intersection with $\{x,y\}$), we
%     have  $y \in F_2$, so that $y \not\in F_1$ (because $y$ is not in $F_1 \cap
%     F_2$). Since $X$ is compact Hausdorff, it is regular, and we can pick
%     disjoint open sets $U$ and $V$ such that $F_1 \subseteq U$, $y \in V$. In
%     particular, $x \in U$, as required.
% \end{proof}


\end{document}
